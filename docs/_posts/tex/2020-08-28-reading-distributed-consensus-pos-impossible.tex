% ---
% title: "[Draft] Reading: Distributed Consensus from Proof of Stake is Impossible"
% date: 2020-08-28
% parent: Posts
% layout: post
% published: true
% ---

\documentclass{article}
\usepackage{footnote}
% \usepackage[utf8]{inputenc}
\usepackage{hyperref}
\usepackage{amsmath}
\usepackage[capitalise,nameinlink,noabbrev]{cleveref}


\newtheorem{idea}{Idea}




\begin{document}

Distributed Consensus from Proof of Stake is Impossible PDF: \href{../../files/pdfs/2014-05-28-distributed-consensus-pos/old-pos.pdf}{old-pos.pdf}, \href{https://xertrov.github.io/fi/files/pdfs/2014-05-28-distributed-consensus-pos/old-pos.pdf}{(permalink)}

The paper \emph{Distributed Consensus from Proof of Stake is Impossible} has been criticised as being hard to read and understand.

My goal is to accurately analyse the paper to extract its arguments, and re-express them in simple language.

% \begin{align}
% x & = y \\
% E & = {mc^2} \\
% \frac{E}{m} & = \frac{mc^2}{m} \\
% \frac{b}{2a} \pm \frac{\sqrt{-4ac}}{2a} & = \frac{Gm_1m_2}{r^2}
% \end{align}

\tableofcontents

\section{Reading: Introduction}

People propose proof-of-stake (PoS) can be used to protect blockchains (i.e. is the consensus mechanism).
The proposition to use proof-of-stake is flawed.
The paper looks at the history and motivation of proof-of-work (PoW) (as used in Bitcoin).
PoW (or its history and motivation?) "evades a impossibility result" [sic].
The paper shows PoS is not a [good] replacement.

\section{Reading: Proof of Stake}

\subsection{Reading: What is Proof of Stake?}

As cryptography has progressed, the idea that [\emph{information} can be real and valuable] is being taken seriously.
People know about economic activity which depends on crypto stuff.
Business stuff can be done on the public internet safely.
People also know about bad things that happen if secret data is lost.

\begin{idea} \label{def:inforeal}
    Information can be real and valuable.
\end{idea}

Since Bitcoin, \cref{def:inforeal} is concrete.
All at once:
\begin{itemize}
    \item You can hold and trade a \emph{fungible} "store of value" (i.e. money).
    \item You can do this using the internet.
    \item You can do this with cryptographic security, instead of physical security.
    \item The security prevents fraud and theft.
\end{itemize}
Instead of "this key is worth \$X because that is the cost of it being public", one can say "this key is worth \$X but is divisible, you can send \$Y to someone but keep the rest".

Proof-of-stake is simple in this\footnote{Which context? I am guessing the context of a public(!) distributed ledger; sending cryptocoins on a public net. The list provided above plus the "instead of \dots one can say \dots" bit.} context. "A proof of stake is a cryptographic proof of ownership."
That proof can prove (precise) ownership of coins. It can also prove that the coins satisfy some property, like being locked up.

Particularly: a proof of stake of some cryptocurrency is like a proof that the hodler will benefit if the projects succeeds. If the stake (from the proof) is time-locked\footnote{Time-locked means that the coins can't be spent by the owner till some point in the future.} then it shows the owner has interest in the project's long term success.

\subsection{Reading: What is distributed consensus?}

\begin{idea}[Distributed Consensus]
    A \emph{distributed consensus} is an agreement between parties. That agreement is global. The parties don't have to trust each-other. The parties don't need identities. The parties didn't need to have been present when the system started.
\end{idea}

A synchronous network is required. You can't assume nodes need to agree on precise timing of events or the order of events. You can't make that assumption for networks over the internet anyway.

For blockchains, we just need distributed consensus on the order of transactions (and nothing else). That means nodes must agree on the first transaction that moves particular funds. That means the recipient can trust that the network treats those coins as his.

The \emph{double-spending problem} is the reason we need distributed consensus. Someone could always try to send the same money to two different people. Both transactions could look valid. Recipients need to know that if there are conflicts about ownership they will still get their money. Distributed consensus means nodes can agree that the one recorded first is the right one.\footnote{The author is a bit unclear b/c of the way they've worded it. They haven't considered the case of two recipients, etc. Personally I think they should have left most of that out; it was unnecessary.}

(Other problems with blockchains are easy and we know how to solve them.)

\subsection{Reading: How does Bitcoin achieve distributed consensus?}

We can prove that distributed consensus cannot be cryptographically guaranteed on an async network. Bitcoin gets around this by only requiring an economic guarantee. It does this via external opportunity cost. That opportunity cost is CPU time and electricity. Bitcoin provides rewards within the system, but only if consensus and an unbroken transaction history is maintained.

Proofs of work in Bitcoin prove an opportunity cost was paid and how much it was. Such a proof includes all work (past proofs) up to that point.\footnote{Particularly proofs of work include all work done in the history \emph{the miner selects}, and not proofs were from other histories.} Bitcoin nodes choose the history with the most total work as the valid history to extend. (Thought it's a bit fuzzy at the head.) This history is called the consensus history. You can only spend coins on the consensus history. So miners are incentivized to all work on the same history. Also, one miner can't control the consensus history without help from their peers.

\subsection{Reading: How is proof of stake used to achieve distributed consensus?}

The big idea behind PoS is move opportunity cost from outside the system to inside it. People want to do this because PoW incentivises ppl to do as much work as possible. Bitcoin work is thermodynamic work. For this kind of work, the Landauer limit lets us figure out what "as much work as possible" means. The result is a network that uses a lot of energy. This drives us towards the heat death of the universe. It drives us "literally as fast as the laws of physics will allow"\footnote{Pfft. No it doesn't. The laws of physics would go way faster. Also the author needs to learn about basic economics. Basic physics too.} If opportunity cost is inside the system we should be able to use less resources (and limit their usage).

PoS works b/c currency holders can lock up their funds ("stake") in some staking scheme. This action is cryptographically verifiable. Particular people publish signed updates to extend the chain's history. Normally a small random selection of people are chosen to do this, and only a majority need to agree to publish an extension. Those people get a reward and can unlock their stake later.\footnote{I think the original paper is missing some details but they're not super important here.}

PoS doesn't depend on the high cost of taking control of the main history (unlike PoW). Instead, PoS says stakeholders will agree on the extension. They'll do this b/c:
\begin{itemize}
    \item they're randomly chosen so are unlikely to collude, and 
    \item they don't want to undermine the system anyway, and
    \item they can't do much damage anyway b/c new random stakeholders will be chosen soon who will probably agree on a single history to extend.
\end{itemize}

\subsection{Reading: What is wrong with this mechanism for consensus?}

PoS equates state to temporarily sacrificed [internal] resources. This begs the question of consensus on who owns what.\footnote{This is because we are trying to answer the question "who owns what" (consensus) by using the last answer of "who owns what?" I think that's what the author means by "begging the question". They should try to be more clear and less fancy.} PoS fans evade this problem with the argument:
\begin{itemize}
    \item False histories must be made by stakeholders, and
    \item Stakeholders' power is limited to a short interval, and
    \item When they have power they're incentivised to behave.
    \item Therefore conflicting histories won't happen, and
    \item the "weak synchronicity" is enough to agree on a single history.
\end{itemize}

The problem with the argument is simple.\footnote{That's a lie.} 

\end{document}
